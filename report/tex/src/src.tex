\section{Общий метод и алгоритм решения}

Для работы с очередями используется ZMQ, программа собирается при помощи Makefile. Управляющий узел – server, вычислительные узлы – client. client и server - отдельные программы, причем новые вычислительные узлы создаются через fork() и execl(...). Для удобной работы с ZMQ создадим удобные обёртки над функциями библиотеки (заголовочный файл m\_zmq.h). Сокеты будем создавать через ipc. В качестве типа сокетов возьмём Publish-Subscribe сокеты как наиболее подходящие. Также напишем классы Socket, Client и Server для работы на высоком уровне. Для хранения дерева общего вида будем использовать собственный класс IdTree.


При запуске программы создаётся сервер (управляющий узел). Затем создаётся вычислительный узел - корень дерева. Сервер работает в двух потоках: первый - работа с пользователем и отправка сообщений, второй - приём сообщений. Endpoint сокета создаётся по pid процесса вычислительного узла. При добавлении и удалении узлов происходит работа с IdTree, это необходимо, чтобы быстро проверять существование узла. Получив сообщение, вычислительный узел может либо отправить его обратно с новыми данными, либо отправить вниз своим детям (это нужно для операции heartbit и корректного уничтожения дерева процессов).

\pagebreak

\section{Исходный код}

\textbf{m\_zmq.h}

\begin{lstlisting}[language=C++]

#pragma once

#include <string>

void* create_zmq_context();
void destroy_zmq_context(void* context);

enum class SocketType {
  PUBLISHER,
  SUBSCRIBER
};
void* create_zmq_socket(void* context, SocketType type);
void close_zmq_socket(void* socket);

enum class EndpointType {
  CHILD_PUB,
  PARRENT_PUB,
};
std::string create_endpoint(EndpointType type, pid_t id);

void bind_zmq_socket(void* socket, std::string endpoint);
void unbind_zmq_socket(void* socket, std::string endpoint);
void connect_zmq_socket(void* socket, std::string endpoint);
void disconnect_zmq_socket(void* socket, std::string endpoint);

enum class CommandType {
  ERROR,
  RETURN,
  CREATE_CHILD,
  REMOVE_CHILD,
  TIMER_TIME,
  TIMER_START,
  TIMER_STOP,
  HEARTBIT
};

struct Message {
  CommandType command = CommandType::ERROR;
  int to_id;
  int value;
  bool go_up = false;
  int uniq_num;

  Message();
  Message(CommandType command_a, int to_id_a, int value_a);
};

bool operator==(const Message& lhs, const Message& rhs);

void send_zmq_msg(void* socket, Message msg);
Message get_zmq_msg(void* socket);

\end{lstlisting}

\textbf{m\_zmq.cpp}

\begin{lstlisting}[language=C++]

#include "m_zmq.h"

#include <errno.h>
#include <string.h>
#include <unistd.h>
#include <zmq.h>

#include <iostream>
#include <tuple>

using namespace std;

void* create_zmq_context() {
  void* context = zmq_ctx_new();
  if (context == NULL) {
    throw runtime_error("Can't create new context. pid:" + to_string(getpid()));
  }
  return context;
}

void destroy_zmq_context(void* context) {
  if (zmq_ctx_destroy(context) != 0) {
    throw runtime_error("Can't destroy context. pid:" + to_string(getpid()));
  }
}

int get_zmq_socket_type(SocketType type) {
  switch (type) {
    case SocketType::PUBLISHER:
      return ZMQ_PUB;
    case SocketType::SUBSCRIBER:
      return ZMQ_SUB;
    default:
      throw logic_error("Undefined socket type");
  }
}

void* create_zmq_socket(void* context, SocketType type) {
  int zmq_type = get_zmq_socket_type(type);
  void* socket = zmq_socket(context, zmq_type);
  if (socket == NULL) {
    throw runtime_error("Can't create socket");
  }
  if (zmq_type == ZMQ_SUB) {
    if (zmq_setsockopt(socket, ZMQ_SUBSCRIBE, 0, 0) == -1) {
      throw runtime_error("Can't set ZMQ_SUBSCRIBE option");
    }
    int linger_period = 0;
    if (zmq_setsockopt(socket, ZMQ_LINGER, &linger_period, sizeof(int)) == -1) {
      throw runtime_error("Can't set ZMQ_LINGER option");
    }
  }
  return socket;
}

void close_zmq_socket(void* socket) {
  //cerr << "closing socket..." << endl;
  sleep(1);  // Don't comment it, because sometimes zmq_close blocks
  if (zmq_close(socket) != 0) {
    throw runtime_error("Can't close socket");
  }
}

string create_endpoint(EndpointType type, pid_t id) {
  switch (type) {
    case EndpointType::PARRENT_PUB:
      return "ipc://tmp/parrent_pub_"s + to_string(id);
    case EndpointType::CHILD_PUB:
      return "ipc://tmp/child_pub_"s + to_string(id);
    default:
      throw logic_error("Undefined endpoint type");
  }
}

void bind_zmq_socket(void* socket, string endpoint) {
  if (zmq_bind(socket, endpoint.data()) != 0) {
    throw runtime_error("Can't bind socket");
  }
}

void unbind_zmq_socket(void* socket, string endpoint) {
  if (zmq_unbind(socket, endpoint.data()) != 0) {
    throw runtime_error("Can't unbind socket");
  }
}

void connect_zmq_socket(void* socket, string endpoint) {
  if (zmq_connect(socket, endpoint.data()) != 0) {
    throw runtime_error("Can't connect socket");
  }
}

void disconnect_zmq_socket(void* socket, string endpoint) {
  if (zmq_disconnect(socket, endpoint.data()) != 0) {
    throw runtime_error("Can't disconnect socket");
  }
}

int counter = 0;
Message::Message() {
  uniq_num = counter++;
}

Message::Message(CommandType command_a, int to_id_a, int value_a)
    : Message() {
  command = command_a;
  to_id = to_id_a;
  value = value_a;
}

bool operator==(const Message& lhs, const Message& rhs) {
  return tie(lhs.command, lhs.to_id, lhs.value, lhs.uniq_num) == tie(rhs.command, rhs.to_id, rhs.value, rhs.uniq_num);
}

void create_zmq_msg(zmq_msg_t* zmq_msg, Message msg) {
  zmq_msg_init_size(zmq_msg, sizeof(Message));
  memcpy(zmq_msg_data(zmq_msg), &msg, sizeof(Message));
}

void send_zmq_msg(void* socket, Message msg) {
  zmq_msg_t zmq_msg;
  create_zmq_msg(&zmq_msg, msg);
  if (!zmq_msg_send(&zmq_msg, socket, 0)) {
    throw runtime_error("Can't send message");
  }
  zmq_msg_close(&zmq_msg);
}

Message get_zmq_msg(void* socket) {
  zmq_msg_t zmq_msg;
  zmq_msg_init(&zmq_msg);
  if (zmq_msg_recv(&zmq_msg, socket, 0) == -1) {
    return Message{CommandType::ERROR, 0, 0};
  }
  Message msg;
  memcpy(&msg, zmq_msg_data(&zmq_msg), sizeof(Message));
  zmq_msg_close(&zmq_msg);
  return msg;
}

\end{lstlisting}

\textbf{socket.h}

\begin{lstlisting}[language=C++]

#pragma once

#include <unistd.h>

#include <memory>
#include <unordered_map>

#include "socket.h"
#include "tree.h"

class Server {
 public:
  Server();
  ~Server();

  pid_t pid() const;
  Message last_message() const;

  void send(Message message);
  Message receive();

  bool check(int id);
  void create_child_cmd(int id, int parrent_id);
  void remove_child_cmd(int id);
  void exec_cmd(int id, CommandType type);
  void heartbit_cmd(int time);
  void print_tree();

  friend void* second_thread(void* serv_arg);

 private:
  pid_t pid_;
  void* context_ = nullptr;
  std::unique_ptr<Socket> publiser_;
  std::unique_ptr<Socket> subscriber_;

  pthread_t receive_msg_loop_id;
  bool terminated_ = false;

  Message last_message_;
  IdTree tree_;
  std::unordered_map<int, bool> map_for_check_;
};

\end{lstlisting}

\textbf{client.h}

\begin{lstlisting}[language=C++]

#pragma once

#include <unistd.h>

#include <chrono>
#include <memory>
#include <string>

#include "socket.h"

class Client {
 public:
  Client(int id, std::string parrent_endpoint);
  ~Client();

  int id() const;
  pid_t pid() const;

  void send_up(Message message);
  void send_down(Message message);
  Message receive();

  void start_timer();
  void stop_timer();
  int get_time();
  void heartbit(int time);

  void
  add_child(int id);

 private:
  int id_;
  pid_t pid_;
  void* context_ = nullptr;
  std::unique_ptr<Socket> child_publiser_;
  std::unique_ptr<Socket> parrent_publiser_;
  std::unique_ptr<Socket> subscriber_;

  bool is_timer_started = false;
  std::chrono::steady_clock::time_point start_;
  std::chrono::steady_clock::time_point finish_;

  bool terminated_ = false;
};

\end{lstlisting}

\textbf{client.cpp}

\begin{lstlisting}[language=C++]

#include "client.h"

#include <unistd.h>

#include <iostream>

#include "m_zmq.h"

using namespace std;

const string CLIENT_EXE = "./client";
const double MESSAGE_WAITING_TIME = 1;

Client::Client(int id, std::string parrent_endpoint) {
  id_ = id;
  pid_ = getpid();
  cerr << to_string(pid_) + " Starting client..."s << endl;
  context_ = create_zmq_context();

  string endpoint = create_endpoint(EndpointType::CHILD_PUB, getpid());
  child_publiser_ = make_unique<Socket>(context_, SocketType::PUBLISHER, endpoint);
  endpoint = create_endpoint(EndpointType::PARRENT_PUB, getpid());
  parrent_publiser_ = make_unique<Socket>(context_, SocketType::PUBLISHER, endpoint);

  subscriber_ = make_unique<Socket>(context_, SocketType::SUBSCRIBER, parrent_endpoint);

  sleep(MESSAGE_WAITING_TIME);
  send_up(Message(CommandType::CREATE_CHILD, id, getpid()));
}

Client::~Client() {
  if (terminated_) {
    cerr << to_string(pid_) + " Client double termination"s << endl;
    return;
  }

  cerr << to_string(pid_) + " Destroying client..."s << endl;
  terminated_ = true;

  sleep(MESSAGE_WAITING_TIME);
  try {
    child_publiser_ = nullptr;
    parrent_publiser_ = nullptr;
    subscriber_ = nullptr;
    destroy_zmq_context(context_);
  } catch (exception& ex) {
    cerr << to_string(pid_) + " " + " Client wasn't destroyed: "s << ex.what() << endl;
  }
}

void Client::send_up(Message message) {
  message.go_up = true;
  parrent_publiser_->send(message);
}

void Client::send_down(Message message) {
  message.go_up = false;
  child_publiser_->send(message);
}

Message Client::receive() {
  Message msg = subscriber_->receive();
  return msg;
}

void Client::add_child(int id) {
  pid_t child_pid = fork();
  if (child_pid == -1) {
    throw runtime_error("Can't fork");
  }
  if (child_pid == 0) {
    execl(CLIENT_EXE.data(), CLIENT_EXE.data(), to_string(id).data(), child_publiser_->endpoint().data(), NULL);
    throw runtime_error("Can't execl ");
  }

  string endpoint = create_endpoint(EndpointType::PARRENT_PUB, child_pid);
  subscriber_->subscribe(endpoint);
}

void Client::start_timer() {
  is_timer_started = true;
  start_ = std::chrono::steady_clock::now();
}

void Client::stop_timer() {
  is_timer_started = false;
  finish_ = std::chrono::steady_clock::now();
}

int Client::get_time() {
  if (is_timer_started) {
    finish_ = std::chrono::steady_clock::now();
  }
  return std::chrono::duration_cast<std::chrono::milliseconds>(finish_ - start_).count();
}

void Client::heartbit(int time) {
  sleep((double)time / 1000);  //s to ms
  send_up(Message(CommandType::HEARTBIT, id_, 0));
}

int Client::id() const {
  return id_;
}

pid_t Client::pid() const {
  return pid_;
}

\end{lstlisting}

\textbf{client\_main.cpp}

\begin{lstlisting}[language=C++]

#include <signal.h>

#include <iostream>
#include <string>

#include "client.h"

using namespace std;

const int ERR_TERMINATED = 1;
const int UNIVERSAL_MESSAGE_ID = -256;

Client* client_ptr = nullptr;
void TerminateByUser(int) {
  if (client_ptr != nullptr) {
    client_ptr->~Client();
  }
  cerr << to_string(getpid()) + " Terminated by user"s << endl;
  exit(0);
}

void process_msg(Client& client, const Message msg) {
  cerr << to_string(getpid()) + " Client message: "s << static_cast<int>(msg.command) << " " << msg.to_id << " " << msg.value << endl;
  switch (msg.command) {
    case CommandType::ERROR:
      throw runtime_error("Error message recieved");
    case CommandType::RETURN:
      client.send_up(msg);
      break;
    case CommandType::CREATE_CHILD:
      client.add_child(msg.value);
      break;
    case CommandType::REMOVE_CHILD: {
      if (msg.to_id != UNIVERSAL_MESSAGE_ID) {
        client.send_up(msg);
      }
      Message tmp = msg;
      tmp.to_id = UNIVERSAL_MESSAGE_ID;
      client.send_down(tmp);
      TerminateByUser(0);
      break;
    }
    case CommandType::TIMER_START:
      client.start_timer();
      client.send_up(msg);
      break;
    case CommandType::TIMER_STOP:
      client.stop_timer();
      client.send_up(msg);
      break;
    case CommandType::TIMER_TIME: {
      int val = client.get_time();
      client.send_up(Message(CommandType::TIMER_TIME, client.id(), val));
      break;
    }
    case CommandType::HEARTBIT:
      client.send_down(msg);
      client.heartbit(msg.value);
      break;
    default:
      throw logic_error("Not implemented message command");
  }
}

int main(int argc, char const* argv[]) {
  if (argc != 3) {
    cerr << argc;
    cerr << "USAGE: " << argv[0] << " <id> <parrent_pub_endpoint" << endl;
  }

  try {
    if (signal(SIGINT, TerminateByUser) == SIG_ERR) {
      throw runtime_error("Can't set SIGINT signal");
    }
    if (signal(SIGSEGV, TerminateByUser) == SIG_ERR) {
      throw runtime_error("Can't set SIGSEGV signal");
    }

    Client client(stoi(argv[1]), string(argv[2]));
    client_ptr = &client;
    cerr << to_string(getpid()) + " Client is started correctly"s << endl;

    while (true) {
      Message msg = client.receive();
      if (msg.to_id != client.id() && msg.to_id != UNIVERSAL_MESSAGE_ID) {
        if (msg.go_up) {
          client.send_up(msg);
        } else {
          client.send_down(msg);
        }
        continue;
      }
      process_msg(client, msg);
    }

  } catch (exception& ex) {
    cerr << to_string(getpid()) + " Client exception: "s << ex.what() << "\nTerminated by exception" << endl;
    exit(ERR_TERMINATED);
  }
  cerr << to_string(getpid()) + " Client is finished correctly"s << endl;
  return 0;
}

\end{lstlisting}

\textbf{server.h}

\begin{lstlisting}[language=C++]

#pragma once

#include <unistd.h>

#include <memory>
#include <unordered_map>

#include "socket.h"
#include "tree.h"

class Server {
 public:
  Server();
  ~Server();

  pid_t pid() const;
  Message last_message() const;

  void send(Message message);
  Message receive();

  bool check(int id);
  void create_child_cmd(int id, int parrent_id);
  void remove_child_cmd(int id);
  void exec_cmd(int id, CommandType type);
  void heartbit_cmd(int time);
  void print_tree();

  friend void* second_thread(void* serv_arg);

 private:
  pid_t pid_;
  void* context_ = nullptr;
  std::unique_ptr<Socket> publiser_;
  std::unique_ptr<Socket> subscriber_;

  pthread_t receive_msg_loop_id;
  bool terminated_ = false;

  Message last_message_;
  IdTree tree_;
  std::unordered_map<int, bool> map_for_check_;
};

\end{lstlisting}

\textbf{server.cpp}

\begin{lstlisting}[language=C++]

#include "server.h"

#include <pthread.h>
#include <signal.h>
#include <unistd.h>

#include <iostream>

#include "m_zmq.h"

using namespace std;

const int ERR_LOOP = 2;
const int ERR_EXEC = 3;
const string CLIENT_EXE = "./client";
const double MESSAGE_WAITING_TIME = 1;
const int UNIVERSAL_MESSAGE_ID = -256;

void* second_thread(void* serv_arg) {
  Server* server_ptr = (Server*)serv_arg;
  pid_t server_pid = server_ptr->pid();
  try {
    pid_t child_pid = fork();
    if (child_pid == -1) {
      throw runtime_error("Can't fork");
    }
    if (child_pid == 0) {
      execl(CLIENT_EXE.data(), CLIENT_EXE.data(), "0", server_ptr->publiser_->endpoint().data(), NULL);
      cerr << "Can't execl "s + CLIENT_EXE << endl;
      kill(server_pid, SIGINT);
      exit(ERR_EXEC);
    }

    string endpoint = create_endpoint(EndpointType::PARRENT_PUB, child_pid);
    server_ptr->subscriber_ = make_unique<Socket>(server_ptr->context_, SocketType::SUBSCRIBER, endpoint);
    server_ptr->tree_.add_to(0, {0, child_pid});

    while (true) {
      Message msg = server_ptr->subscriber_->receive();
      if (msg.command == CommandType::ERROR) {
        if (server_ptr->terminated_) {
          return NULL;
        } else {
          cerr << "This bom" << endl;
          throw runtime_error("Can't receive message");
        }
      }
      server_ptr->last_message_ = msg;
      cerr << "Message on server: " << static_cast<int>(msg.command) << " " << msg.to_id << " " << msg.value << endl;

      switch (msg.command) {
        case CommandType::CREATE_CHILD: {
          auto& pa = server_ptr->tree_.get(msg.to_id);
          pa.second = msg.value;
          cout << "OK: " << server_ptr->last_message_.value << endl;
          break;
        }
        case CommandType::REMOVE_CHILD:
          server_ptr->tree_.remove(msg.to_id);
          cout << "OK" << endl;
          break;
        case CommandType::TIMER_START:
          cout << "OK:" << msg.to_id << endl;
          break;
        case CommandType::TIMER_STOP:
          cout << "OK:" << msg.to_id << endl;
          break;
        case CommandType::TIMER_TIME: {
          cout << "OK:" << msg.to_id << ": " << msg.value << endl;
          break;
        }
        case CommandType::HEARTBIT:
          server_ptr->map_for_check_[msg.to_id] = true;
          break;
        default:
          break;
      }
    }

  } catch (exception& ex) {
    cerr << "Server exctption: " << ex.what() << "\nTerminated by exception on server receive loop" << endl;
    exit(ERR_LOOP);
  }
  return NULL;
}

Server::Server() {
  pid_ = getpid();
  cerr << to_string(pid_) + " Starting server..."s << endl;
  context_ = create_zmq_context();

  string endpoint = create_endpoint(EndpointType::CHILD_PUB, getpid());
  publiser_ = make_unique<Socket>(context_, SocketType::PUBLISHER, endpoint);

  if (pthread_create(&receive_msg_loop_id, 0, second_thread, this) != 0) {
    throw runtime_error("Can't run second thread");
  }
}

Server::~Server() {
  if (terminated_) {
    cerr << to_string(pid_) + " Server double termination" << endl;
    return;
  }

  cerr << to_string(pid_) + " Destroying server..."s << endl;
  terminated_ = true;

  for (pid_t pid : tree_.get_all_second()) {
    kill(pid, SIGINT);
  }

  try {
    publiser_ = nullptr;
    subscriber_ = nullptr;
    destroy_zmq_context(context_);
  } catch (exception& ex) {
    cerr << "Server wasn't destroyed: " << ex.what() << endl;
  }
}

void Server::send(Message message) {
  message.go_up = false;
  publiser_->send(message);
}

Message Server::receive() {
  return subscriber_->receive();
}

pid_t Server::pid() const {
  return pid_;
}

Message Server::last_message() const {
  return last_message_;
}

bool Server::check(int id) {
  Message msg(CommandType::RETURN, id, 0);
  send(msg);
  sleep(MESSAGE_WAITING_TIME);
  if (last_message_ == msg) {
    return true;
  } else {
    return false;
  }
}

void Server::create_child_cmd(int id, int parrent_id) {
  if (tree_.find(id)) {
    cout << "Error: Already exists" << endl;
    return;
  }
  if (!tree_.find(parrent_id)) {
    cout << "Error: Parent not found" << endl;
    return;
  }
  if (!check(parrent_id)) {
    cout << "Error: Parent is unavailable" << endl;
    return;
  }
  send(Message(CommandType::CREATE_CHILD, parrent_id, id));
  tree_.add_to(parrent_id, {id, 0});
}

void Server::remove_child_cmd(int id) {
  if (id == 0) {
    cout << "Can't remove zero child" << endl;
    return;
  }
  if (!tree_.find(id)) {
    cout << "Error: Not found" << endl;
    return;
  }
  if (!check(id)) {
    cout << "Error: Node is unavailable" << endl;
    return;
  }
  send(Message(CommandType::REMOVE_CHILD, id, 0));
}

void Server::exec_cmd(int id, CommandType type) {
  if (!tree_.find(id)) {
    cout << "Error: Not found" << endl;
    return;
  }
  if (!check(id)) {
    cout << "Error: Node is unavailable" << endl;
    return;
  }
  send(Message(type, id, 0));
}

void Server::heartbit_cmd(int time) {
  if (time < 1000) {
    cout << "Too low time for heartbit" << endl;
  }
  send(Message(CommandType::HEARTBIT, UNIVERSAL_MESSAGE_ID, time));
  auto uset = tree_.get_all_first();
  for (int id : uset) {
    map_for_check_[id] = false;
  }
  sleep(4 * (double)time / 1000);
  for (auto& [id, bit] : map_for_check_) {
    if (!bit) {
      cout << "Heartbit: node " << id << " is unavailable now" << endl;
    }
    bit = false;
  }
  cout << "OK" << endl;
}

void Server::print_tree() {
  tree_.print();
}

\end{lstlisting}

\textbf{server\_main.cpp}

\begin{lstlisting}[language=C++]

#include <signal.h>

#include <iostream>
#include <string>

#include "server.h"

using namespace std;

const int ERR_TERMINATED = 1;

Server* server_ptr = nullptr;
void TerminateByUser(int) {
  if (server_ptr != nullptr) {
    server_ptr->~Server();
  }
  cerr << to_string(getpid()) + " Terminated by user"s << endl;
  exit(0);
}

void process_cmd(Server& server, string cmd) {
  if (cmd == "check") {
    int id;
    cin >> id;
    if (server.check(id)) {
      cout << "OK" << endl;
    } else {
      cout << "NOT_OK" << endl;
    }
  } else if (cmd == "create") {
    int id, parrent_id;
    cin >> id >> parrent_id;
    server.create_child_cmd(id, parrent_id);
  } else if (cmd == "remove") {
    int id;
    cin >> id;
    server.remove_child_cmd(id);
  } else if (cmd == "exec") {
    int id;
    string sub_cmd;
    cin >> id >> sub_cmd;
    CommandType type;
    if (sub_cmd == "time") {
      type = CommandType::TIMER_TIME;
    } else if (sub_cmd == "start") {
      type = CommandType::TIMER_START;
    } else if (sub_cmd == "stop") {
      type = CommandType::TIMER_STOP;
    } else {
      cout << "Incorrect subcommand" << endl;
      return;
    }
    server.exec_cmd(id, type);
  } else if (cmd == "heartbit") {
    int time;
    cin >> time;
    server.heartbit_cmd(time);
  } else if (cmd == "print_tree") {
    server.print_tree();
  } else {
    cout << "It's not a command" << endl;
  }
}

int main() {
  try {
    if (signal(SIGINT, TerminateByUser) == SIG_ERR) {
      throw runtime_error("Can't set SIGINT signal");
    }
    if (signal(SIGSEGV, TerminateByUser) == SIG_ERR) {
      throw runtime_error("Can't set SIGSEGV signal");
    }

    Server server;
    server_ptr = &server;
    cerr << to_string(getpid()) + " Server is started correctly"s << endl;

    string cmd;
    while (cin >> cmd) {
      process_cmd(server, cmd);
    }

  } catch (exception& ex) {
    cerr << to_string(getpid()) + " Server exception: "s << ex.what() << "\nTerminated by exception" << endl;
    exit(ERR_TERMINATED);
  }
  cerr << to_string(getpid()) + " Server is finished correctly"s << endl;
  return 0;
}


\end{lstlisting}

\pagebreak