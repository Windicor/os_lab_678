\section{Постановка задачи}

{\bfseries Цель работы:} 

Приобретение практических навыков в:

\begin{itemize}
    \item Управлении серверами сообщений (№6)
    \item Применение отложенных вычислений (№7)
    \item Интеграция программных систем друг с другом (№8)
\end{itemize}

{\bfseries Задание (вариант 21):} 

Реализовать распределенную систему по асинхронной обработке запросов. В данной
распределенной системе должно существовать 2 вида узлов: «управляющий» и
«вычислительный». Необходимо объединить данные узлы в соответствии с той топологией,
которая определена вариантом. Связь между узлами необходимо осуществить при помощи
технологии очередей сообщений. Также в данной системе необходимо предусмотреть проверку
доступности узлов в соответствии с вариантом. При убийстве («kill -9») любого вычислительного
узла система должна пытаться максимально сохранять свою работоспособность, а именно все
дочерние узлы убитого узла могут стать недоступными, но родительские узлы должны сохранить
свою работоспособность.

Управляющий узел отвечает за ввод команд от пользователя и отправку этих команд на
вычислительные узлы. Список основных поддерживаемых команд:

\textbf{Создание нового вычислительного узла}\\
Формат команды: create id [parent]\\
id – целочисленный идентификатор нового вычислительного узла\\
parent – целочисленный идентификатор родительского узла. Если топологией не предусмотрено
введение данного параметра, то его необходимо игнорировать (если его ввели)\\
Формат вывода:\\
«Ok: pid», где pid – идентификатор процесса для созданного вычислительного узла\\
«Error: Already exists» - вычислительный узел с таким идентификатором уже существует\\
«Error: Parent not found» - нет такого родительского узла с таким идентификатором\\
«Error: Parent is unavailable» - родительский узел существует, но по каким-то причинам с ним не
удается связаться
«Error: [Custom error]» - любая другая обрабатываемая ошибка\\
\textit{Примечания: создание нового управляющего узла осуществляется пользователем программы
при помощи запуска исполняемого файла. Id и pid — это разные идентификаторы.
}\\


\textbf{Удаление существующего вычислительного узла}\\
Формат команды: remove id\\
id – целочисленный идентификатор удаляемого вычислительного узла
Формат вывода:\\
«Ok» - успешное удаление\\
«Error: Not found» - вычислительный узел с таким идентификатором не найден\\
«Error: Node is unavailable» - по каким-то причинам не удается связаться с вычислительным узлом\\
«Error: [Custom error]» - любая другая обрабатываемая ошибка\\
\textit{Примечание: при удалении узла из топологии его процесс должен быть завершен и
работоспособность вычислительной сети не должна быть нарушена.
}\\

\textbf{Исполнение команды на вычислительном узле}\\
Формат команды: exec id [params]\\
id – целочисленный идентификатор вычислительного узла, на который отправляется команда\\
Формат вывода:\\
«Ok:id: [result]», где result – результат выполненной команды\\
«Error:id: Not found» - вычислительный узел с таким идентификатором не найден\\
«Error:id: Node is unavailable» - по каким-то причинам не удается связаться с вычислительным
узлом\\
«Error:id: [Custom error]» - любая другая обрабатываемая ошибка\\
\textit{Примечание: выполнение команд должно быть асинхронным. Т.е. пока выполняется команда на
одном из вычислительных узлов, то можно отправить следующую команду на другой
вычислительный узел.}\\

\textbf{Вариант 21.}\\ 
\textbf{Топология 2}\\
Все вычислительные узлы находятся в дереве общего вида. Есть только один управляющий узел. \\

\textbf{Набор команд 3 (локальный таймер)}\\
Формат команды сохранения значения: exec id subcommand\\
subcommand – одна из трех команд: start, stop, time.\\
start – запустить таймер\\
stop – остановить таймер\\
time – показать время локального таймера в миллисекундах\\
Пример:

\begin{verbatim}

> exec 10 time
Ok:10: 0
>exec 10 start
Ok:10
>exec 10 start
Ok:10
*прошло 10 секунд*
> exec 10 time
Ok:10: 10000
*прошло 2 секунды*
>exec 10 stop
Ok:10
*прошло 2 секунды*
>exec 10 time
Ok:10: 12000

\end{verbatim}



\textbf{Команда проверки 3}\\
Формат команды: heartbit time\\
Каждый узел начинает сообщать раз в time миллисекунд о том, что он работоспособен. Если от узла нет сигнала в течении 4*time миллисекунд, то должна выводится пользователю строка: «Heartbit: node id is unavailable now», где id – идентификатор недоступного вычислительного узла.

Пример:

\begin{verbatim}
    
> heartbit 2000
Ok

\end{verbatim}

\pagebreak
