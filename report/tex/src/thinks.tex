\section{Вывод}

В процессе написания лабораторной я научился основам работы с серверами сообщений на примере ZMQ, изучил возможные подходы к созданию приложений на основе серверов сообщений. Технология действительно удобна для решения ряда задач, а конкретно ZMQ обладает достаточно высокой скоростью передачи данных. Конечно, у ZMQ есть ряд недостатков, но особых трудностей при написании лабораторной это не вызвало. Использование C++ оказалось оправданным в этой работе, так как это позволило быстро разработать ряд классов-абстракций и не держать в голове архитектуру целиком. Также удобными оказаль механизм исключений и наличие деструкторов классов (не требоволась рассматривать все случаи, в которых, например, надо закрывать сокет и т. п.) Задача показалась достаточной интересной и необычной, поэтому решать её было удовольствием. 

